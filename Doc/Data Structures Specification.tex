\documentclass[a4paper,10pt]{article}

\usepackage[utf8]{inputenc}
\usepackage[american]{babel}
\usepackage{fontenc}
\usepackage{graphicx}
\usepackage{verbatimbox}
\usepackage{hyperref}
\usepackage{paralist}
\usepackage{enumerate}
\usepackage{amsmath}
\usepackage{enumitem}
\usepackage{gitinfo2}
\usepackage{xifthen}

\newcommand{\parenthesize}[2]{{\em  #1\ifthenelse{\isempty{#2}}{}{(#2)}}}
\newcommand{\longProduct}{Contacts Discovery}
\newcommand{\shortProduct}{CD}
\newcommand{\commercialName}{CD}
\newcommand{\client}[1][]{\parenthesize{Client}{#1}}
\newcommand{\server}{{\em Server}}
\newcommand{\user}[1][]{\parenthesize{User}{#1}}
\newcommand{\Server}{{\em \MakeUppercase{server}}}
\newcommand{\Client}[1][]{\parenthesize{\MakeUppercase{client}}{#1}}
\newcommand{\User}[1][]{\parenthesize{\MakeUppercase{user}}{#1}}

\author{\gitAuthorName{}\\\gitAuthorEmail{}}
\date{\gitAuthorDate{}\\Version \gitAbbrevHash{} \gitVtag{}}

\let\Item\item
\newcommand\SpecialItem{\renewcommand\item[1][]{\Item[\textbullet~\bfseries##1]}
}
\renewcommand\enddescription{\endlist\global\let\item\Item}


\title{\longProduct{} System Data Structures Specification}

\newcommand{\common}{{\em Common}}

\begin{document}

\maketitle
\tableofcontents

\section{Purpose}
This document specifies the core data structures that are used by the \shortProduct{} software implementation. It covers the entities \client{}~\ref{sec:structure:client} and 
\server{}~\ref{sec:structure:server}. Structures common to more than one entity are under~\ref{sec:structure:common} and they are referred under entity \common{}.

\section{References}
For the project overview, see~\cite{project_spec}. For the flows and protocols specification, see~\cite{flow_spec}.

\section{Definitions, Nomenclature and Conventions}

\subsection{Portability}
The data structures and their fields are specified in a way that is independent from the programming language and the execution platform. For integer types, the width is specified 
to avoid platform-dependent implementations from diverging. String and composite types (maps, hashtables, etc) are SDK-dependent, but implementation should not rely on any uncommon 
and SDK-specific behavior (eg, string can correctly be interpreted as an STL or Java string, but no type requirement beyond that is intended).

\subsection{Attributes}
Data structures are marked with a set of attributes. A list of markers in all caps appears after the data structure. These attributes are:

\newcommand{\persistent}{\MakeUppercase{persistent}}
\newcommand{\ephemeral}{\MakeUppercase{ephemeral}}
\newcommand{\secure}{\MakeUppercase{secure}}
\newcommand{\unsecure}{\MakeUppercase{unsecure}}
\newcommand{\infrastructure}{\MakeUppercase{infrastructure}}
\newcommand{\unique}{\MakeUppercase{unique}}

\begin{description}
 \item[\persistent] The data structure persists process restarts. For the client, this translates to the structure stored in the application storage space. For the server, that it 
is stored in the database or the filesystem.
 \item[\ephemeral] The opposite of \persistent. The data structure remains only in memory.
 \item[\secure] The data structure is to be stored in secure storage. Such data structures cannot leak or be tampered with. For the client, it is stored preferably in an encrypted 
lightweight database, such as SQLCipher~\cite{sql_cipher}. For the server, it means stored in a database that is secured and its installation hardened. \secure{} normally implies 
\persistent, although it is in theory possible that \ephemeral{} structures are secured in memory.
 \item[\unsecure] The data structure is not stored in secure storage. Such data structures can leak or be tampered without loss of information or compromising operation. An example 
is UI settings.
 \item[\infrastructure] Infrastructure data structures are used to compose higher level structures. They do not appear independently, therefore the infrastructure attribute is 
expected to appear alone.
 \item[\unique] This data structure is expected to have only one instantiation in each process. {\em Attention: this does not mean that the structure is to be implemented as a 
singleton}. Treat this attribute as informative and only as informative. This specification may change at any time due to requirement changes.
\end{description}

\subsection{Comments}
After the specification of the data structure, a short explanation of its functionality and usage may follow. This explanation should be treated as informative and not as part of 
the specification.

\subsection{Naming}
A data structure $X$ for entity $Y$ is specified in a section named after $Y$. When referring to $X$ from another section, its name is $Y.X$. Field $Z$ within $X$ is referred to 
as $X.Z$ ($Y.X.Z$ from another entity section).

All type names are upper CamelCase. All field names are lower camelCase.

\subsubsection{Non-native Primitive Types}
There are some field types, for example RSA keys, that are non-native, but are used in the specification as native. We assume that such types come from a library and their 
specification is implied by the library's API. As long as their intended functionality is well understood, we do not specify them further. If some points need to be elaborated on, 
details are provided in the comments.

\subsubsection{Containers and Enums}
Containers, such as maps and lists, are denoted with their properties. For example, a FIFO queue is denoted as such, although the platform implementation may be something like 
$deque$ in STL. We denote the types of their parameters inside parentheses, separated by colons. Therefore, a map from type $A$ to type $B$ is denoted as $map(A:B)$.

Enumerations are denoted as $\{ A, B, \dots \}$ where $A, B, \dots$ are members of the enumeration. Enumeration members are written in upper CamelCase.

\subsubsection{Inheritance, Typedefs and Templates}
A data structure $B$ extending $A$ is denoted as

\begin{verbbox}
B extends A
\end{verbbox}
\begin{center}
\theverbbox
\end{center}

When a structure $B$ is a typedef of $A$, it is denoted as

\begin{verbbox}
B = A
\end{verbbox}
\begin{center}
\theverbbox
\end{center}

A template class $A$ with template parameters $B$ and $C$ is denoted as

\begin{verbbox}
A<B, C>
\end{verbbox}
\begin{center}
\theverbbox
\end{center}

For programming languages that do not support template constructs, inheritance can be used to achieve the same effect.

\subsection{Database}

\subsection{Keys, Indexes, Primary and Foreign Keys}
We define where applicable what the primary key is for a structure, which fields are foreign keys, which are keys and which have indexes. If the structure has a field named $id$, 
this is implicitly the primary key. If a structure has a primary key that is not named $id$, we explicitly define it.

\subsubsection{State}
Some of the operations on server structures involve manipulating state. These operations must be performed in the database, not the 
server code, which is stateless. If the database does not have the appropriate mechanisms to execute them, in the manner of MySQL triggers (\cite{mysql_trigger}), then some 
creative solution, like a garbage collection process, must be implemented.

\subsection{Initial Values and Default Values}
Where applicable we define where the initial value for a field comes from and what the default value is when the source is allowed to not provide one.

\subsection{Time}
All time-related fields are specified as having type {\em time}. This should be interpreted as a UNIX timestamp or any equivalent structure that carries full timezone information. 
All timestamps are generated at the server, for consistency across the system. In case the server is distributed, all server-side processes (including the DB) must share a common 
clock.

\section{Specification}

\subsection{Common}
\label{sec:structure:common}

\subsubsection{UserAgent}

\begin{verbbox}
UserAgent
{
  platform, string
  version, string
}
\end{verbbox}
\begin{center}
\theverbbox
\end{center}

\begin{inparaitem}[ ]
 \item \infrastructure
\end{inparaitem}

\paragraph*{Comments}
This structure is a representation of the type of client. It is built at process startup.

\SpecialItem
\begin{description}
 \item[platform] The platform information given by the system. This usually contains OS, OS version, hardware information, etc.
 \item[version] The version of the user agent. The type may change to something more structured.
\end{description}

\subsubsection{Id}

\begin{verbbox}
Id = 64-bit signed integer
\end{verbbox}
\begin{center}
\theverbbox
\end{center}

\begin{inparaitem}[ ]
 \item \infrastructure
\end{inparaitem}

\paragraph*{Comments}
Type alias for id types. It is defined as a signed data type instead of unsigned due to the unsigned type not being supported by most languages and systems.

\subsubsection{Client}

\begin{verbbox}
Client extends UserAgent
{
  id, Id
}
\end{verbbox}
\begin{center}
\theverbbox
\end{center}

\begin{inparaitem}[ ]
 \item \infrastructure
\end{inparaitem}

\paragraph*{Comments}
This structure extends a user agent with a unique id across a user. The $id$ field is not a primary key for the system, it is a primary key inside a User clients set
(see~\ref{sec:structure:server:user}).

\subsubsection{User}

\begin{verbbox}
User
{
  id, Id
}
\end{verbbox}
\begin{center}
\theverbbox
\end{center}

\begin{inparaitem}[ ]
 \item \infrastructure
\end{inparaitem}

\paragraph*{Comments}
This is the base for any type of User. It is a simple wrapper around an id, which acts as the key of any subtype.

\subsubsection{ClientId}

\begin{verbbox}
ClientId
{
  user, Id
  client, Id
}
\end{verbbox}
\begin{center}
\theverbbox
\end{center}

\begin{inparaitem}[ ]
 \item \infrastructure
\end{inparaitem}

\paragraph*{Comments}
Convenience data structure that pairs together a user and client id. It is used to identify a client in the entire system.

\subsubsection{TimeoutMap}

\begin{verbbox}
TimeoutMap<T>
{
  timeout, 16-bit unsigned integer,
    initial value from configuration, default = 60 000 (1'')
  Map = map(T:time)
  entries, Map
  entriesList, FIFO queue(Map.iterator)
}
\end{verbbox}
\begin{center}
\theverbbox
\end{center}

\begin{inparaitem}[ ]
 \item \infrastructure
\end{inparaitem}

\paragraph*{Comments}
This is a structure whose implementation depends a lot on the data storage back-end. On RDBMSs it is straightforward to implement it, but NoSQL backends can only implement a 
simplified version of it, where the garbage-collection is done by the server. The following is specified with RDBMSs in mind. For NoSQL the specification is used to define the 
functionality of the garbage collector.

This structure is a map for expirable objects and for periodically garbage-collecting items that have expired. The garbage collection can be triggered periodically (the period 
should not be less than two times the timeout value) or when the map size has exceeded some threshold.

Note that entries are removed from the map only during garbage collection. If a lookup operation retrieves an expired entry, it is not removed from the map, because the 
corresponding iterator needs to be removed from the list. This can be implemented, but it would complicate the code and it would degrade performance, for minimal gain in memory 
consumption.

\SpecialItem
\begin{description}
 \item[timeout] The time in msecs for which an entry is considered active. If a platform offers a type for time difference, like C++11 does, this more appropriate type must be 
used for the field.
 \item[entries] It maps objects of type $T$ to a timestamp produced when an object is inserted into the map. If hash maps are available for the platform, they must be used instead 
of a map.
 \item[entriesList] This list is used to garbage-collect items efficiently. The entries point to items in the orders map. The FIFO nature of the queue ensures that garbage 
collection can start from the oldest entry and proceed until an unexpired entry is found. The cost of garbage collection is constant per entry. Please note that if a platform 
doesn't support iterators, the entry id can be used as the value type.
\end{description}

\subsubsection{HashAlgorithm}

\begin{verbbox}
HashAlgorithm = { SHA-256 }
\end{verbbox}
\begin{center}
\theverbbox
\end{center}

\begin{inparaitem}[ ]
 \item \infrastructure
\end{inparaitem}

\paragraph*{Comments}
HashAlgorithm is an enum of all the acceptable hash algorithms in the system. For the moment only SHA-256 is accepted.

\subsubsection{HashBuffer}

\begin{verbbox}
HashBuffer
{
  buffer, byte array
  algorithm, HashAlgorithm
}
\end{verbbox}
\begin{center}
\theverbbox
\end{center}

\begin{inparaitem}[ ]
 \item \infrastructure
\end{inparaitem}

\paragraph*{Comments}
A HashBuffer is a hash with the algorithm used to produce it. The algorithm accompanies the buffer for forward-compatibility.

\SpecialItem
\begin{description}
 \item[buffer] A buffer with the hash. The buffer is unencoded.
 \item[algorithm] Normally SHA-256.
\end{description}

\SpecialItem
\begin{description}
 \item[Primary Key] All fields.
\end{description}

\subsubsection{EncryptionAlgorithm}
\begin{verbbox}
EncryptionAlgorithm = { RSA-2048 }
\end{verbbox}
\begin{center}
\theverbbox
\end{center}

\begin{inparaitem}[ ]
 \item \infrastructure
\end{inparaitem}

\paragraph*{Comments}
EncryptionAlgorithm is an enum of all the acceptable encryption key generation algorithms in the system. For the moment only RSA-2048 is accepted.

\subsubsection{EncryptedBuffer}

\begin{verbbox}
EncryptedBuffer
{
  buffer, byte array
  algorithm, EncryptionAlgorithm
  client, ClientId
}
\end{verbbox}
\begin{center}
\theverbbox
\end{center}

\begin{inparaitem}[ ]
 \item \infrastructure
\end{inparaitem}

\paragraph*{Comments}
An EncryptedBuffer is a ciphertext payload with the algorithm used to produce it and the client that generated it. The algorithm accompanies the buffer for forward-compatibility.

\SpecialItem
\begin{description}
 \item[buffer] A buffer with the ciphertext.
 \item[algorithm] The encryption algorithm. It should be the same as that of the key used to produce it.
 \item[client] The owner client of the encryption key.
\end{description}

\begin{description}
 \item[Primary Key] Buffer and algorithm.
\end{description}

\subsubsection{SignatureAlgorithm}
\begin{verbbox}
SignatureAlgorithm = { ECDSA-prime192v1 }
\end{verbbox}
\begin{center}
\theverbbox
\end{center}

\begin{inparaitem}[ ]
 \item \infrastructure
\end{inparaitem}

\paragraph*{Comments}
SignatureAlgorithm is an enum of all the acceptable signature key generation algorithms in the system. For the moment only ECDSA-prime192v1 is accepted.

\subsubsection{SignatureBuffer}

\begin{verbbox}
SignatureBuffer
{
  buffer, byte array
  algorithm, SignatureAlgorithm
  client, ClientId
}
\end{verbbox}
\begin{center}
\theverbbox
\end{center}

\begin{inparaitem}[ ]
 \item \infrastructure
\end{inparaitem}

\paragraph*{Comments}
A SignatureBuffer is a ciphertext payload with the algorithm used to produce it and the client that generated it. The algorithm accompanies the buffer for forward-compatibility.

\SpecialItem
\begin{description}
 \item[buffer] A buffer with the ciphertext.
 \item[algorithm] The signature algorithm. It should be the same as that of the key used to produce it.
 \item[client] The owner client of the signing key.
\end{description}

\begin{description}
 \item[Primary Key] Buffer and algorithm.
\end{description}

\subsubsection{EncryptionKey}

\begin{verbbox}
EncryptionKey
{
  buffer, byte array with X.509 encoding of the key
  algorithm, EncryptionAlgorithm
}
\end{verbbox}
\begin{center}
\theverbbox
\end{center}

\begin{inparaitem}[ ]
 \item \infrastructure
\end{inparaitem}

\paragraph*{Comments}
An EncryptionKey is a buffer with the X.509 encoding of the key with the algorithm used to produce it. The algorithm accompanies the buffer for forward-compatibility.

\subsubsection{VerificationKey}

\begin{verbbox}
PublicKey
{
  buffer, byte array with X.509 encoding of the key
  algorithm, SignatureAlgorithm
}
\end{verbbox}
\begin{center}
\theverbbox
\end{center}

\begin{inparaitem}[ ]
 \item \infrastructure
\end{inparaitem}

\paragraph*{Comments}
A VerificationKey is a buffer with the X.509 encoding of the key with the algorithm used to produce it. The algorithm accompanies the buffer for forward-compatibility.

\subsubsection{PublicKeys}

\begin{verbbox}
PublicKeys
{
  encryption, EncryptionKey
  verification, VerificationKey
}
\end{verbbox}
\begin{center}
\theverbbox
\end{center}

\begin{inparaitem}[ ]
 \item \infrastructure
\end{inparaitem}

\SpecialItem
\begin{description}
 \item[encryption] The encryption key.
 \item[verification] The signature verification key.
\end{description}

\subsubsection{ContactOperationType}

\begin{verbbox}
ContactOperationType = { Add, Delete, Update }
\end{verbbox}
\begin{center}
\theverbbox
\end{center}

\begin{inparaitem}[ ]
 \item \infrastructure
\end{inparaitem}

\paragraph*{Comments}
The kind of contact operation requested by the client.

\subsubsection{ContactOperation}

\begin{verbbox}
ContactOperation
{
  contact, HashBuffer
  type, ContactOperationType
  comparisonIdentity, EncryptedBuffer
}
\end{verbbox}
\begin{center}
\theverbbox
\end{center}

\begin{inparaitem}[ ]
 \item \infrastructure
\end{inparaitem}

\SpecialItem\paragraph*{Comments}
This structure is used for the protocol between client and server.

\begin{description}
 \item[contact] The blinded contact.
 \item[type] The type of operation.
\end{description}

\subsubsection{ContactOperationSet}

\begin{verbbox}
ContactOperation = set(ContactOperation)
\end{verbbox}
\begin{center}
\theverbbox
\end{center}

\begin{inparaitem}[ ]
 \item \unsecure
\end{inparaitem}

\subsubsection{RandomHashPad}

\begin{verbbox}
RandomHashPad = 128-bit signed integer
\end{verbbox}
\begin{center}
\theverbbox
\end{center}

\begin{inparaitem}[ ]
 \item \infrastructure
\end{inparaitem}

\paragraph*{Comments}
This is a type definition for the randomness added to hashed data sent to \Server{}.

\subsection{Client}
\label{sec:structure:client}

\subsubsection{ContactType}
\begin{verbbox}
ContactType = { MSISDN, Email, Facebook, Skype, Twitter, Address,
                Other }
\end{verbbox}
\begin{center}
\theverbbox
\end{center}

\begin{inparaitem}[ ]
 \item \infrastructure
\end{inparaitem}

\paragraph*{Comments}
ContactType is an enum of all the possible types that the system knows. $Other$ is the default type.

\subsubsection{Identity}

\begin{verbbox}
Identity
{
  target, string
  type, ContactType
  blinded, HashBuffer
}
\end{verbbox}
\begin{center}
\theverbbox
\end{center}

\begin{inparaitem}[ ]
 \item \infrastructure
\end{inparaitem}

\paragraph*{Comments}
This is a central structure that represents a simple contact. The field $blinded$ is empty only for contacts that have not been registered with the service.

\SpecialItem
\begin{description}
 \item[target] We assume that the target is normalized and validated by the platform and it is given in its normalized form from the platform SDK. If it turns out that this is not 
the case for a platform and contact type, the client will apply normalization before importing contacts of this type.
 \item[type] The type reported by the platform is matched to one of the members of ContactType.
 \item[blinded] The value of this field is empty for all contacts for which there is no known registered identity.
\end{description}

\subsubsection{RegistrationLink}

\begin{verbbox}
RegistrationLink = URL
(
  protocol = 'kareebo'
  host = 'registration'
  path = an encoding of Id and RandomHashPad
)
\end{verbbox}
\begin{center}
\theverbbox
\end{center}

\begin{inparaitem}[ ]
 \item \infrastructure
\end{inparaitem}

\paragraph*{Comments}
A RegistrationLink is a URL whose protocol is handled by the client and whose path encodes the user id of the client that requests registration and its identity. In total the path 
encodes 192 bits.

\subsubsection{InvitationRequestLink}

\begin{verbbox}
InvitationRequestLink = URL
(
  protocol = 'kareebo'
  host = 'invitation'
  path = an encoding of Identity
)
\end{verbbox}
\begin{center}
\theverbbox
\end{center}

\begin{inparaitem}[ ]
 \item \infrastructure
\end{inparaitem}

\paragraph*{Comments}
An InvitationRequestLink is a URL whose protocol is handled by the client and whose path encodes an identity, such an MSISDN, an e-mail etc.

\subsubsection{DecryptionKey}

\begin{verbbox}
DecryptionKey
{
  buffer, byte array with PKCS#8 encoding of the key
  algorithm, EncryptionAlgorithm
}
\end{verbbox}
\begin{center}
\theverbbox
\end{center}

\begin{inparaitem}[ ]
 \item \infrastructure
\end{inparaitem}

\paragraph*{Comments}
A DecryptionKey is a buffer with the PKCS\#8 encoding of the key with the algorithm used to produce it. The algorithm accompanies the buffer for forward-compatibility.

\subsubsection{SigningKey}

\begin{verbbox}
SigningKey
{
  buffer, byte array with PKCS#8 encoding of the key
  algorithm, SignatureAlgorithm
}
\end{verbbox}
\begin{center}
\theverbbox
\end{center}

\begin{inparaitem}[ ]
 \item \infrastructure
\end{inparaitem}

\paragraph*{Comments}
A SigningKey is a buffer with the PKCS\#8 encoding of the key with the algorithm used to produce it. The algorithm accompanies the buffer for forward-compatibility.

\subsubsection{PrivateKeys}
\label{sec:structure:client:private_keys}

\begin{verbbox}
PrivateKeys
{
  decryption, DecryptionKey
  signing, SigningKey
}
\end{verbbox}
\begin{center}
\theverbbox
\end{center}

\begin{inparaitem}[ ]
 \item \infrastructure
\end{inparaitem}

\SpecialItem
\begin{description}
 \item[decryption] The decryption key.
 \item[signing] The signing key.
\end{description}

\subsubsection{User}

\begin{verbbox}
User extends Common.User
{
  client, Client
  keys, PrivateKeys
}
\end{verbbox}
\begin{center}
\theverbbox
\end{center}

\begin{inparaitem}[ ]
 \item \unique
 \item \secure
 \item \persistent
\end{inparaitem}

\paragraph*{Comments}
This data structure holds the information related to the customer using the application.

\SpecialItem
\begin{description}
 \item[client] The id of the client and the user agent information.
 \item[keys] Two pairs of elliptic curve encryption and signing keys are generated by the client at first use. The public keys are sent to the server, as they are used to by other 
clients and the server for encrypting their contact information and verifying the signature of the client. The private keys are kept by the client.
\end{description}

\subsubsection{Contact}
\label{sec:structure:client:contact}

\begin{verbbox}
Contact
{
  identity, Identity
  attributes, list((string,string))
}
\end{verbbox}
\begin{center}
\theverbbox
\end{center}

\begin{inparaitem}[ ]
 \item \infrastructure
\end{inparaitem}

\paragraph*{Comments}
This extends the $Identity$ structure, with a set of attributes that come from the native contact book and are not relevant for the service.

\SpecialItem
\begin{description}
 \item[identity] The $Identity$ structure of the contact.
 \item[attributes] A list of key-value pairs. These are transported from client to client and are inserted to the receiver's contact book by a contact adaptor that runs in the 
context of the receiving client. This way if the contacts representation for one platform changes, backwards compatibility is not a problem and the adaptors of other platforms can 
be updated to cover the changes. The attributes are containing essentially any optional contact information.
\end{description}

\subsubsection{PersonalName}

\begin{verbbox}
PersonalName
{
  first, string
  middle, string
  last, string
  title, string
}
\end{verbbox}
\begin{center}
\theverbbox
\end{center}

\begin{inparaitem}[ ]
 \item \infrastructure
\end{inparaitem}

\paragraph*{Comments}
This data structure represents a personal name that is part of a contact card. All fields may be empty, in which case the UI must surface a contact as display name. The rules 
for what is being displayed are specified in~\cite{flow_spec}.

\subsubsection{ContactCard}

\begin{verbbox}
ContactCard
{
  id, Id
  name, PersonalName
  contacts, set(Contact)
}
\end{verbbox}
\begin{center}
\theverbbox
\end{center}

\begin{inparaitem}[ ]
 \item \infrastructure
\end{inparaitem}

\SpecialItem
\begin{description}
 \item[id] This is generated by the client and it is a serial number.
 \item[name] It comes from the platform contact book.
 \item[contacts] It comes from the platform contact book.
\end{description}

\subsubsection{ContactBook}

\begin{verbbox}
ContactBook = set(ContactCard)
\end{verbbox}
\begin{center}
\theverbbox
\end{center}

\begin{inparaitem}[ ]
 \item \persistent
 \item \secure
 \item \unique
\end{inparaitem}

\subsubsection{UserContactBook}

\begin{verbbox}
ContactBook = ContactCard
\end{verbbox}
\begin{center}
\theverbbox
\end{center}

\begin{inparaitem}[ ]
 \item \persistent
 \item \secure
 \item \unique
\end{inparaitem}

\subsubsection{PendingIdentity}

\begin{verbbox}
PendingIdentity
{
  identity, Identity
  blind, RandomHashPad
}
\end{verbbox}
\begin{center}
\theverbbox
\end{center}

\begin{inparaitem}[ ]
 \item \infrastructure
\end{inparaitem}

\paragraph*{Comments}
This holds information about a pending identity registration request.

\SpecialItem
\begin{description}
 \item[identity] The information about the identity whose registration is pending.
 \item[blind] The blind used to generate $identity$.
\end{description}

\SpecialItem
\begin{description}
 \item[Primary Key] All fields.
\end{description}

\subsubsection{PendingIdentitySet}

\begin{verbbox}
PendingIdentitySet = TimeoutMap<PendingIdentity>
\end{verbbox}
\begin{center}
\theverbbox
\end{center}

\begin{inparaitem}[ ]
 \item \unique
 \item \secure
 \item \persistent
\end{inparaitem}

\paragraph*{Comments}
The map removes pending identity registration information after they timeout without confirmation. The timeout is set to a year, so that the mappings are not removed prematurely. 
Since the map is not expected to grow large enough to create space problems, the map could be implemented as FIFO queue with a maximum size, as long as the size is large enough 
entries are not cleared prematurely.

\SpecialItem
\begin{description}
 \item[identity] The information about the identity whose registration is pending.
 \item[blind] The blind used to generate $identity$.
\end{description}

\subsubsection{Blind}

\begin{verbbox}
Blind = RandomHashPad
\end{verbbox}
\begin{center}
\theverbbox
\end{center}

\begin{inparaitem}[ ]
 \item \unique
 \item \secure
 \item \persistent
\end{inparaitem}

\paragraph*{Comments}
This value can be set only once during the lifetime of a client. If it is overwritten or lost after it has been set, the client needs to reset its state and start as if installed 
for the first time.

\subsection{Server}
\label{sec:structure:server}

\subsubsection{Client}

\begin{verbbox}
Client extends Common.Client
{
  keys, PublicKeys
  comparisonIdentities, set(EncryptedBuffer)
}
\end{verbbox}
\begin{center}
\theverbbox
\end{center}

\begin{inparaitem}[ ]
 \item \infrastructure
\end{inparaitem}

\paragraph*{Comments}

The structure extends a client description with a field to indicate whether this client has been registered or its registration is pending and can be garbage-collected.

\SpecialItem
\begin{description}
 \item[keys] The public keys of the client.
 \item[comparisonIdentities] The set of encrypted identities used for matching identities between different client contact books.
\end{description}

\subsubsection{User}
\label{sec:structure:server:user}

\begin{verbbox}
User
{
  id, Id
  clients, set(Client)
  identities, set(HashBuffer)
  blind, RandomHashPad
  sentRequests, set(HashBuffer)
}
\end{verbbox}
\begin{center}
\theverbbox
\end{center}

\begin{inparaitem}[ ]
 \item \infrastructure
\end{inparaitem}

\paragraph*{Comments}
This data structure holds the information related to a user and its registered clients.

\SpecialItem
\begin{description}
 \item[id] This number uniquely identifies the user across the system. It is a primary key in the user's table in the server DB. The server generates this id.
 \item[clients] A set of $Client$.
 \item[identities] A set of blinded identities.
 \item[blind] Populated during the first identity registration.
 \item[sentRequests] The set of contact requests already sent.
\end{description}

\subsubsection{UserSet}

\begin{verbbox}
UserSet = set(User)
\end{verbbox}
\begin{center}
\theverbbox
\end{center}

\begin{inparaitem}[ ]
 \item \secure
 \item \persistent
 \item \unique
\end{inparaitem}

\paragraph*{Comments}
A set of $User$ objects, indexed by $User.id$. This is the global user table and entry point for retrieving user related information.

\subsubsection{HashIdentity}

\begin{verbbox}
HashIdentity
{
  id, Id
  confirmers, set(Id)
}
\end{verbbox}
\begin{center}
\theverbbox
\end{center}

\begin{inparaitem}[ ]
 \item \infrastructure
\end{inparaitem}

\paragraph*{Comments}
This is the value type of \ref{sec:structure:server:hash_identity_map}. It is the id of the User that owns this identity and the set of Users that have confirmed it.

\SpecialItem
\begin{description}
 \item[id] The $User.id$ of the owner of the identity.
 \item[confirmers] The set of $User$s that have confirmed this identity.
\end{description}

\SpecialItem
\begin{description}
 \item[Foreign Key] All ids are foreign keys in $UserSet$.
\end{description}

\subsubsection{HashIdentityMap}
\label{sec:structure:server:hash_identity_map}
\begin{verbbox}
HashIdentityMap = map(HashBuffer:HashIdentity)
\end{verbbox}
\begin{center}
\theverbbox
\end{center}

\begin{inparaitem}[ ]
 \item \secure
 \item \persistent
 \item \unique
\end{inparaitem}

\paragraph*{Comments}
A map from blinded identities to $HashIdentity$ objects.

\bibliography{spec,dataStructures}
\bibliographystyle{plain}

\end{document}
