\documentclass[a4paper,10pt]{article}

\newcommand{\yannis}{Yannis Ioannidis}
\newcommand{\longProduct}{Contacts Discovery}
\newcommand{\shortProduct}{CD}
\newcommand{\client}{{\em Client}}
\newcommand{\server}{{\em Server}}


\newcommand{\signedData}{\emph{signedData}}
\newcommand{\encryptedData}{\emph{encryptedData}}
\newcommand{\newPublicKeys}{\emph{newPublicKeys}}
\newcommand{\newPrivateKeys}{\emph{newPrivateKeys}}
\newcommand{\privateKeys}{\emph{privateKeys}}
\newcommand{\publicKeys}{\emph{publicKeys}}
\newcommand{\signingKey}{\privateKeys{}.\emph{signing}}
\newcommand{\verificationKey}{\publicKeys{}.\emph{verification}}
\newcommand{\encryptionKey}{\publicKeys{}.\emph{encryption}}
\newcommand{\decryptionKey}{\privateKeys{}.\emph{decryption}}
\newcommand{\userId}[1]{\identity{#1}{\em.user}}
\newcommand{\userIdSimple}[1]{\parenthesize{userId}{#1}}
\newcommand{\clientIdSimple}[1]{\parenthesize{clientId}{#1}}
\newcommand{\clientId}[1]{\identity{#1}{\em.client}}
\newcommand{\contactOperationSet}{\emph{contactOperationSet}}
\newcommand{\identity}[1]{\parenthesize{identity}{#1}}
\newcommand{\link}[1]{\parenthesize{link}{#1}}
\newcommand{\addedContacts}{{\em addedContacts}}
\newcommand{\deletedContacts}{{\em deletedContacts}}
\newcommand{\updatedContacts}{{\em updatedContacts}}
\newcommand{\encryptionKeyMap}[1]{\parenthesize{encryptionKeyMap}{#1}}
\newcommand{\contactCard}[1]{\parenthesize{contactCard}{#1}}

\title{\longProduct{} System Flows and Protocols Specification}

\begin{document}

\maketitle
\tableofcontents

\section{Purpose}
This document specifies the protocols and flows executed for the \longProduct{} system (\shortProduct{}) between the \server{} and \client{} entities. The protocols cover contact 
management and administration flows (account creation, account deletion, etc).

\section{References}
For the project overview, see \cite{project_spec}. For the supporting data structures, see \cite{data_structure_spec}. For details about security and cryptography, 
see \cite{crypto_spec}. For the client UI elements specification, see \cite{UI_spec}.

\section{Definitions, Nomenclature and Conventions}

\subsection{Entities}
We refer to three entities in this document. These represent different participants in a flow. A short explanation on each one follows. For more detail on their role in the 
system, see \cite{project_spec}. Entities are denoted in all uppercase.

\begin{description}
 \item[\Server{}] includes the central business logic, the database and all interfaces to third-party systems for, among others, sending out notifications and mail.
 \item[\Client{A}] is an abstraction of a user agent of user $A$. It usually refers to the mobile app that a customer uses, but it can extend to desktop applications, test clients, etc.
 \item[\User{A}] This is an entity that refers to the human using \Client{A}. It is used for flow steps that require interaction between the client user agent and the human user. 
This entity is not used for protocol specifications. It is used in flow specifications to disambiguate the origin and target of some steps.
\end{description}

\subsection{Notation for Variables}
\label{sec:variable_notation}
Because in each flow or protocol multiple participants are involved, variables are denoted with a hierarchical name, with each level denoted as a dot. For example, variable $X$ of 
entity $Y$ is denoted as $Y.X$. Inputs and outputs are considered variables.

Variables are declared as
\begin{center}
$\langle Entity\rangle.\langle Name\rangle:\langle type\rangle\{[qualification_1, qualification_2, \dots]\}$.
\end{center}

For example, let $bar$ be an unsigned integer variable of entity $foo$, that takes values in the range $[10,100]$ and has default value 5. The declaration would look like:

\begin{center}
 foo.bar:unsigned integer[range:[10,100],default value:5].
\end{center}

The qualifications part is optional and its syntax is that of a $key:value$ pair.

A data structure $D$ may appear without prior definition, when it has been specified in \cite{data_structure_spec}. For specifications of types related to security and 
cryptography, such as registration codes and elliptic curve keys, see \cite{crypto_spec}.

All variable names are small camelCase.

\subsection{Flow Specifications Organization and Format}\label{sec:flow_spec_def}
Each flow description is organized in the following sections:

\begin{description}
 \item[Description] What the flow does.
 \item[Participants] A list of participating entities.
 \item[Inputs] A list of the inputs to the flow. See \ref{sec:variable_notation} for the variable syntax.
 \item[Outputs] The output of flows is normally success or failure and it can usually be inferred implicitly. We note here interesting state changes for the involved entities.
 \item[Other variables] Any other variables that are instantiated during the flow.
 \item[Flow] An enumeration of the steps. The first word in each step is the initiating entity, meaning the entity that sends a message in this step.
\end{description}

\subsection{Protocol Specifications Organization and Format}

\subsection{Signatures}\label{sec:signatures}

Signing a message $m$ involves signing a hash of $m$ and accompanying the signature with $m$. This means that when \Client{A} signs a message $m$ and sends the signature 
$s$ to \Server{}, \Server{} knows from $s$ the details of the message, like the client id and the payload. See \cite{crypto_spec} for more details.

\subsection{Alternative Steps}
Some parts of the flow may have alternative specifications. These alternatives originate from usability concerns or uncertainty about hardware components, which may require 
multiple ways to effect the same action. Alternative steps are defined after the primary specification.

Alternative flow steps should be considered as future extensions. Initial implementation follows the primary flow specification, but it should take into account that alternative 
steps may replace the primary ones or that extensions of the implementation may incorporate multiple alternatives. Details about implementation priority and plans for these 
alternatives can be found in accompanying footnotes.

\section{Flows}

\subsection{Implicit Flow Branches}
Some flow branches are repeated and are followed under common triggers. To avoid cluttering the main flows, we define them here.

\subsubsection{Flow Termination with Error on Failed Commit}
Steps which state that entity $X$ commits $Y$ to the DB imply a flow branch that is followed when the commit operation fails. In this case, $X$ terminates the flow with an error. 
In most cases, error flows are inferable from other error branches. When this is not possible, the branch is declared explicitly. 

\subsubsection{Server Errors}
All flows involving \Server{} and another entity may fail for various reasons, such as a client being unknown or the signature verification failing. For security, 
\Server{} does not reveal the specific reason the operation failed. Therefore, when, for example, a signature verification fails, the message from \Server{} to \Client{A} 
does not differentiate between all the reasons the verification may fail.

\subsection{Register Identity}
\label{register_identity_flow}
This flow is executed when \user{A} wants to enter the system. This is done by requesting a \user{B} who is already in the system to confirm \user{A}'s identity.

\paragraph{Participants} \User{A}, \User{B}, \Client{A}, \Client{B}, \Server{}

\paragraph{Inputs}
\SpecialItem
\begin{description}
 \item[\identity{A} : Client.Identity] From \Client{A}. This may come from the contact book of \User{A} or by \User{A} manually entering the value.
 \item[\identity{B} : Client.Identity] From the contact book of \User{A}.
\end{description}

\paragraph{Outputs}
For \Client{A} either success, if \Server{} accepts the registration, error otherwise. In case of error, \User{A} is notified. \Client{B} removes the temporary mappings for 
\identity{A} after a certain period without confirmation from \Server{} has passed, as part of a separate flow.

\paragraph{Other Variables}
\SpecialItem
\begin{description}
 \item[\link{A} : RegistrationLink] Constructed by \Client{B} and sent to \User{A}.
 \item[\link{B} : InvitationRequestLink] Constructed by \Client{A} and sent to \User{B}.
 \item[\parenthesize{N}{A} : RandomHashPad] A random blind for \identity{A}.
 \item[\parenthesize{U}{A} : HashBuffer] The blinded \identity{A} generated by \Client{B}.
 \item[\userIdSimple{A} : Id] A user id for \identity{A} generated by \Server{}.
 \item[\clientIdSimple{A} : Id] A client id for \identity{A} generated by \Server{}.
 \item[\parenthesize{$N_S$}{A} : RandomHashPad] A random blind for \identity{A} generated by \Server{}.
 \item[\parenthesize{X}{A} : RandomHashPad] The XOR of two random blinds.
 \item[\privateKeys{} : PrivateKeys] The new private keys for \Client{A}.
 \item[\publicKeys{} : PublicKeys] The new public keys for \Client{B}.
 \item[\parenthesize{$N_C$}{A} : RandomHashPad]\label{register_identity_flow:NCA} A random blind for \identity{A} generated by \Client{A}.
 \item[\signedData{} : SignatureBuffer] The ECDSA signature with \signingKey{} of \Client{A} registration data. See \cite{crypto_spec} for details.
 \item[j: 64-bit signed integer] An index in a set. Normally, it can be a much smaller integer, but for simplicity, we define it with the same type as Id.
 \item[\parenthesize{userAgent}{A} : UserAgent] The user agent of \Client{A}.
\end{description}

\paragraph{Flow}

\begin{enumerate}
 \item \User{A} requests \Client{A} to notify \User{B}.
 \item \Client{A} constructs \link{B} from \identity{A}.
 \item \Client{A} sends \link{B} to \identity{B}.
 \item \User{B} clicks on \link{B} and the flow passes to \Client{B}. If \User{B} doesn't react in this step, the flow terminates.
 \item \Client{B} extracts \identity{A} from \link{B}.
 \item If \Client{B} knows \identity{A} already, it populates $U(A)$ from its local storage and it goes to step~\ref{identity_registration_flow:send_hash}.
 \item \Client{B} generates $N(A)$.
 \item \Client{B} calculates $U(A)=Hash($\identity{A}$,N(A))$.
 \item \Client{B} stores locally the triplet $\langle$\identity{A}$,U(A),N(A)\rangle$ to the PendingIdentitiesSet.
 \item\label{identity_registration_flow:send_hash} \Client{B} sends to \Server{} request for registering $U(A)$.
 \item \Server{} checks in HashIdentityMap if there is a mapping for $U(A)$.
 \begin{enumerate}
  \item If there is, it populates \userIdSimple{A} with the value of the mapping.
  \item If there isn't, it generates a new \userIdSimple{A} and creates a mapping from $U(A)$ to \userIdSimple{A} in HashIdentityMap. It creates an entry in UserSet with 
\userIdSimple{A} and adds $U(A)$ to the UserSet.Identities for \userIdSimple{A}.
 \end{enumerate}
 \item \Server{} adds \userId{B} to HashIdentityMap.Confirmers of $U(A)$.
 \item \Server{} populates $N_S(A)$ and sets UserSet.Blind for \userIdSimple{A} to it.
 \item \Server{} sends to \Client{B} \userIdSimple{A} and $N_S(A)$.
 \item \Client{B} calculates $X(A)=N(A)\oplus N_S(A)$, where $\oplus$ denotes the XOR operation.
 \item \Client{B} generates \link{A} from \userIdSimple{A} and $X(A)$.
 \item \Client{B} sends \link{A} to \identity{A}.
 \item \Client{A} retrieves \userIdSimple{A} and $X(A)$ from \link{A}.
 \item \Client{A} sends to \Server{} \userIdSimple{A}.
 \item \Server{} retrieves $N_S(A)$ from UserSet.Blind and key \userIdSimple{A}.
 \begin{enumerate}
  \item If the retrieval fails, \Server{} sends failure to \Client{A} and terminates the flow.
 \end{enumerate}
 \item \Server{} creates a new client in UserSet for \userIdSimple{A} with id \clientIdSimple{A}.
 \item \Server{} sends $N_S(A)$ and \clientIdSimple{A} to \Client{A}.
 \item \Client{A} calculates $N(A)=N_S(A)\oplus X(A)$.
 \item \Client{A} calculates $U(A)=Hash($\identity{A}$,N(A))$.
 \item \Client{A} populates $N_C(A)$, \privateKeys{} and \publicKeys{}.
 \item \Client{A} stores $N_C(A)$ to local storage as Blind.
 \item \Client{A} stores \privateKeys{} to local storage.
 \item For every identity $i$ that \Client{A} wants to register with \Server{}, including \identity{A}, it generates $U_i(A)=Hash(i,N_C(A))$.
 \item \Client{A} populates \signedData{} from \signingKey{} and \publicKeys{}, \parenthesize{userAgent}{A} from local storage, $U(A)$, \userIdSimple{A}, $\lbrace U_1(A),\ldots,U_n(A)\rbrace$, $j$, where $j$ is the 
index for \identity{A} in the set of identities.
 \item \Client{A} sends to \Server{} \signedData{}, \publicKeys{}, \parenthesize{userAgent}{A}, $U(A)$, \userIdSimple{A}, $\lbrace U_1(A),\ldots,U_n(A)\rbrace$, $j$.
 \item \Server{} verifies the signature and that $U(A)$ maps to \userIdSimple{A} in HashIdentityMap.
 \begin{enumerate}
  \item If the verification fails, \Server{} sends failure to \Client{A} and terminates the flow.
 \end{enumerate}
 \item \Server{} registers $\lbrace U_1(A),\ldots,U_n(A)\rbrace$ for \userIdSimple{A} in HashIdentityMap and UserSet.
 \item \Server{} clears the mapped value for $U(A)$ in HashIdentityMap and marks it as an alias of $U_j(A)$. It sets the confirmer set for $U_j(A)$ to the previous confirmer set 
for $U(A)$.
 \item \Server{} sets public keys and user agent for \Client{A} to \publicKeys{} and \parenthesize{userAgent}{A}.
 \item \Server{} replies with success to \Client{A}.
 \item \Server{} sends to \Client{B} confirmation that $U(A)$ is registered.
 \item \Client{B} executes the Send Contact Card flow \ref{send_contact_card_flow} with \Client{A}.
 \item \Client{B} removes from PendingIdentitiesSet $U(A)$ and adds \identity{A} to the \longProduct{} contact book.
 \item \Client{B} executes the Broadcast New Contact Identity flow \ref{broadcast_new_contact_identity_flow} for \identity{A}, for all of its contacts.
\end{enumerate}

\subsection{Register Unconfirmed Identity}
\label{register_unconfirmed_identity_flow}
This flow registers an unconfirmed identity with \server{}.

\paragraph{Participants} \Client{A}, \User{A}, \Server{}

\paragraph{Inputs}
\SpecialItem
\begin{description}
 \item[\identity{} : Client.Identity] From \Client{A}. This may come from the contact book of \User{A} or by \User{A} manually entering the value.
\end{description}

\paragraph{Outputs}
For \Client{A} either success, if \Server{} register the identity, error otherwise.

\paragraph{Other Variables}
\SpecialItem
\begin{description}
 \item[N : RandomHashPad] A random blind for \identity{}. The same as $N_C(A)$ in \ref{register_identity_flow:NCA}.
 \item[$U_N$ : HashBuffer] The primary blinded \identity{} generated by \Client{A}.
 \item[\privateKeys{} : PrivateKeys] The client private keys.
 \item[\signedData{} : SignatureBuffer] The ECDSA signature with \signingKey{} of $U_R$. See \cite{crypto_spec} for details.
\end{description}

\paragraph{Flow}

\begin{enumerate}
 \item \User{A} enters \identity{} in contact card.
 \item \Client{A} detects that \identity{} has been added.
 \item \Client{A} asks \User{A} if \identity{} should be registered.
 \item \User{A} answers the request.
 \begin{enumerate}
  \item If \User{A} answers no, the flow terminates.
 \end{enumerate}
 \item \Client{A} populates $N$ from Blind in local storage.
 \begin{enumerate}
  \item If \Client{A} doesn't have one in local storage, the flow terminate with error.
 \end{enumerate}
 \item \Client{A} calculates $U_N=Hash($\identity{}$,N)$.
 \item \Client{A} calculates \signedData{} from \signingKey{} and $U_N$.
 \item \Client{A} sends to \Server{} \signedData{}.
 \item \Server{} verifies the signature.
 \item \Server{} stores $U_N$ to UserSet.Identities for \userId{A}.
 \item \Server{} stores $U_N$ to HashIdentityMap mapped to \userId{A}, \clientId{A}.
 \item \Server{} replies to \Client{A} with success.
\end{enumerate}

\paragraph{Implementation Notes} The above flow is described for one identity, but it should be implemented to accept as input a list of identities.
The reason for accepting a list of identities is that a user may enter multiple identities in the contact card and there is no implementation complexity in doing all of them 
together.

\subsection{Broadcast New Contact Identity}
\label{broadcast_new_contact_identity_flow}
This flow is triggered when a client is notified about a new identity from one of its contacts, as, for example, at the end of \ref{register_identity_flow} or at the end of this 
flow.

\paragraph{Participants} \Client{A}, all clients for \userId{B}, \Server{}

\paragraph{Inputs}
\SpecialItem
\begin{description}
 \item[\identity{C} : Client.Identity] From \Client{A}. This is the identity that is being broadcast.
 \item[\parenthesize{U}{C} : HashBuffer] The blinded \identity{C} \Client{A} knows.
\end{description}

\paragraph{Outputs}
If \Client{B} has \identity{C} in its contact book, it receives $U(C)$, and \Server{} connects \identity{C} with \Client{B}.

\paragraph{Other Variables}
\SpecialItem
\begin{description}
 \item[\identity{A} : ClientId] The identity of \Client{A}, known by \Server{}.
 \item[\identity{B} : ClientId] The identity of \Client{B}, known by \Server{}.
 \item[\privateKeys{} : PrivateKeys] The set of private keys.
 \item[\encryptedData{} : EncryptedBuffer] Encrypted payload exchanged between \Client{A} and \Client{B}. See \cite{crypto_spec} for details.
 \item[\signedData{} : SignatureBuffer] The ECDSA signature of \encryptedData{} with \signingKey{}. See \cite{crypto_spec} for details.
 \item[\publicKeys{} : PublicKeys] The set of public keys.
 \item[R : RandomHashPad] A random blind for \identity{C}.
 \item[\encryptionKeyMap{B}] A map from client ids to encryption keys for \userId{B}. Generated by \Server{}.
 \item[I : EncryptedBuffer] A double encryption of \identity{C}
 \item[$I_R$ : EncryptedBuffer] A double encryption of $R$.
 \item[\parenthesize{U'}{C} : HashBuffer] A blinded identity for \identity{C} that is already known by another client.
\end{description}

\paragraph{Flow}

\begin{enumerate}
 \item \Client{A} populates \privateKeys{} from local storage.
 \item \Client{A} calculates \signedData{} from \signingKey{} and \userId{B}.
 \item \Client{A} sends to \Server{} request for contact comparison for \userId{B}, with \signedData{}.
 \item \Server{} populates \encryptionKeyMap{B}.
 \item \Server{} sends to \Client{A} \encryptionKeyMap{B}.
 \item \Client{A} populates \privateKeys{} from local storage.
 \item \Client{A} generates $R$.
\end{enumerate}
The following is executed for each member of \encryptionKeyMap{B} and each \Client{B} corresponding to this member. Each communication round between \Client{A} and \Server{} 
aggregates the values for each run of the rest of the flow.
\begin{enumerate}[resume]
 \item \Client{A} calculates $I=$\identity{C}$^{-1}$ for \encryptionKey{} and sets $I=E(A,E(B,i\odot R))$. The inverse is calculated according to the homomorphism of the crypto 
system, denoted with $\odot$. For RSA, for example, it is multiplication.
 \item \Client{A} calculates $I_R=E(A,E(B,R))$.
 \item \Client{A} populates \signedData{} from \signingKey{} and $I$ and $I_R$.
 \item \Client{A} sends \signedData{} to \Server{}.
 \item \Server{} populates \publicKeys{} from UserSet and \userId{A} and \clientId{A}.
 \item \Server{} verifies \signedData{}.
 \item \Server{} retrieves ComparisonIdentities from UserSet for \userId{B} and \clientId{B} and for each member $i$
 \begin{enumerate}
  \item It calculates $i\odot I$ and compares it to $I_R$. If they are equal the contacts match.
 \end{enumerate}
 \item If there is a match, \Server{} populates \publicKeys{} from \identity{B} and replies to \Client{A} with \identity{B}, \encryptionKey{}, otherwise the flow terminates.
\end{enumerate}
Steps \ref{broadcast_new_contact_identity_flow:one}-\ref{broadcast_new_contact_identity_flow:two} can be executed when the first match is found and their results stored for the 
entire flow.
\begin{enumerate}[resume]
 \item\label{broadcast_new_contact_identity_flow:one} \Client{A} populates \privateKeys{} from local storage.
 \item \Client{A} calculates \encryptedData{} from \encryptionKey{} and \identity{C} and $U(C)$.
 \item\label{broadcast_new_contact_identity_flow:two} \Client{A} calculates \signedData{} from \signingKey{} and \encryptedData{}.
 \item \Client{A} sends to \Server{} \signedData{}, \identity{B}.
 \item \Server{} populates \publicKeys{} from \identity{A}.
 \item \Server{} verifies \signedData{}.
 \item \Server{} sends to \Client{B} \verificationKey{}, \signedData{}.
 \item \Client{B} verifies \signedData{}.
 \item \Client{B} retrieves \identity{C} and $U(C)$ from \encryptedData{}.
 \item \Client{B} adds \identity{C} to contact book with blinded identity $U(C)$.
 \item If \Client{B} already knows \identity{C} with another blinded identity.
 \begin{enumerate}
  \item \Client{B} populates $U'(C)$ with the known blinded identity.
  \item \Client{B} populates \privateKeys{} from local storage.
  \item \Client{B} populates \signedData{} from \signingKey{} and $U(C)$ and $U'(C)$.
  \item \Client{B} sends \signedData{} to \Server{}.
  \item \Server{} populates \publicKeys{} from UserSet for \identity{B}.
  \item \Server{} verifies \signedData{}.
  \item \Server{} marks $U'(C)$ as an alias of $U(C)$ in HashIdentityMap and merges the confirmers of the former into the confirmers of the latter.
  \begin{enumerate}
    \item Both entries are aliases to other entries. In this case, the merge is executed for the pointed to entries and all point to the remaining non-alias entry.
    \item Only one entry is an alias. In this case, the merging happens between the pointed to entry and the non-alias entry of the initial two.
    \item None of the entries are aliases. The one with the smaller set of confirmers becomes an alias. In case of a draw, the one that comes first in a lexicographical sort 
becomes an alias.
  \end {enumerate}
 \end{enumerate}
 \item If the mapping doesn't exist already, \Client{B} executes the Broadcast New Contact Identity \ref{broadcast_new_contact_identity_flow} flow for all its contacts.
 \item \Client{B} triggers the Send Contact Card \ref{send_contact_card_flow} with \User{C}.
\end{enumerate}

\subsection{Modify Keys}
\label{modify_keys_flow}
This flow is triggered periodically from the client. For details about when it is triggered see \cite{crypto_spec}.

\paragraph{Participants} \Client{A}, \Server{}

\paragraph{Inputs}
\SpecialItem
\begin{description}
 \item[\identity{A} : ClientId] From \Client{A} local storage.
 \item[\privateKeys{} : PrivateKeys] The private keys that are valid at the start of the flow. From \Client{A} local storage.
\end{description}

\paragraph{Outputs}
For \Client{A} either success, if \Server{} applied the changes, error otherwise. In case of error, the new keys are discarded and the flow is repeated according to the 
specification in \cite{crypto_spec}.

\paragraph{Other Variables}
\SpecialItem
\begin{description}
 \item[\newPrivateKeys{} : PrivateKeys] The new private keys.
 \item[\newPublicKeys{} : PublicKeys] The new public keys.
 \item[\signedData{} : SignatureBuffer] The ECDSA signature of \newPublicKeys{} with \signingKey{}. See \cite{crypto_spec} for details.
 \item[\publicKeys{} : PublicKeys] The public keys of \Client{A} stored in the DB of \Server{}.
\end{description}

\paragraph{Flow}

\begin{enumerate}
 \item \Client{A} constructs \newPrivateKeys{}, \newPublicKeys{}.
 \item \Client{A} generates \signedData{}.
 \item \Client{A} sends \signedData{}, \newPublicKeys{}, \userId{A} and \clientId{A} to \Server{}.
 \item \Server{} retrieves \publicKeys{} based on \userId{A} and \clientId{A}.
 \item \Server{} verifies the signature of \signedData{}.
 \begin{enumerate}
  \item If the above step fails, \Server{} replies with error.
 \end{enumerate}
 \item \Server{} verifies the integrity of \newPublicKeys{}.
 \begin{enumerate}
  \item If the above step fails, \Server{} replies with error.
 \end{enumerate}
 \item \Server{} commits \newPublicKeys{} to the DB.
 \item \Server{} replies with success.
 \item\label{modify_keys_flow:local_storage} \Client{A} commits \newPrivateKeys{} to secure local storage.
\end{enumerate}

\subsection{Modify User Agent}
\label{modify_user_agent_flow}
This flow is triggered when the user agent has changed, due to client or OS updates. The flow is identical to \ref{modify_keys_flow}, with the difference that \signedData{} 
signs the new user agent, instead of \newPublicKeys{}. The implementation must be common, parameterized for the input data.

\subsection{Update Server Contact Book}
\label{update_server_bontact_book_flow}
This flow updates the contact book that is stored server-side. For new contacts, it adds their comparison identity and adds an edge to the graph, for deleted contacts, it removes their comparison identity and deletes an edge from the graph, and or updated contacts, it adds and edge to the graph.

The difference between an add and an update operation is that an add operation may be executed before the client know the blinded identity of the added contact, while an update opertion adds the blinded identity for a contact that has been added without one.

\paragraph{Participants} \Client{A}, \Server{}

\paragraph{Inputs}
\SpecialItem
\begin{description}
 \item[\identity{A} : ClientId] From \Client{A} local storage.
 \item[\addedContacts{} : set(Contact)] The set of added contacts. This set results from contact book operations.
 \item[\deletedContacts{} : set(Contact)] The set of deleted contacts. This set results from contact book operations.
 \item[\updatedContacts{} : set(Contact)] The set of contacts for which a hash has become known since added.
\end{description}

\paragraph{Outputs}
For \Client{A} either success, if \Server{} updated all contacts successfully, failure if something went wrong during the update. In case of failure the entire set is rejected.

\paragraph{Other Variables}
\SpecialItem
\begin{description}
 \item[\publicKeys{} : PublicKeys] The public keys of \Client{A}.
 \item[\privateKeys{} : PrivateKeys] The private keys of \Client{A}.
 \item[\signedData{} : SignatureBuffer] \contactOperationSet{} signed with \privateKeys{}. See \cite{crypto_spec} for details.
 \item[\contactOperationSet{} : ContactOperationSet] The set of contact operations to be performed by \Server{}.
\end{description}

\paragraph{Flow}

\begin{enumerate}
 \item \Client{A} generates \contactOperationSet{}.
 \item \Client{A} populates \privateKeys{} from local storage.
 \begin{enumerate}
  \item For each $c$ in \addedContacts{}, \Client{A} adds to \contactOperationSet{} a ContactOperation with type $Add$, contact $c.hash$, and comparison identity $E(A,c.target)$. 
Note that $c.hash$ may be empty for those contacts for which the hash is not known yet.
  \item For each $c$ in \deletedContacts{}, \Client{A} adds to \contactOperationSet{} a ContactOperation with type $Delete$, contact $c.hash$, and comparison identity 
$E(A,c.target)$. Note that $c.hash$ may be empty for those contacts for which the hash is not known yet.
  \item For each $c$ in \updatedContacts{}, \Client{A} adds to \contactOperationSet{} a ContactOperation with type $Update$, contact $c.hash$, and empty comparison identity. Note 
that $c.hash$ cannot be empty for these contacts, since the update operation is executed only for these contacts for which the hash became known through 
\ref{broadcast_new_contact_identity_flow}.
 \end{enumerate}
 \item \Client{A} generates \signedData{} from \signingKey{} and \contactOperationSet{}.
 \item \Client{A} sends \signedData{} and \contactOperationSet{} to \Server{}.
 \item \Server{} retrieves \publicKeys{} from its DB based on \userId{A} and \clientId{A}.
 \item \Server{} verifies \signedData{} based on \contactOperationSet{}.
 \item For each entry $a$ in \contactOperationSet{}
 \begin{enumerate}
  \item If the operation is {\em Add}
  \begin{enumerate}
   \item \Server{} adds $a.comparisonIdentity$ to UserSet.ComparisonIdentities for \userId{A} and \clientId{A}. If the entry exists, the flow terminates with error.
   \item\label{update_server_bontact_book_flow:one} \Server{} updates the graph and generates the appropriate notifications to \Client{A}.
  \end{enumerate}
  \item If the operation is {\em Delete}
  \begin{enumerate}
   \item \Server{} deletes $a.comparisonIdentity$ from UserSet.ComparisonIdentities for \userId{A} and \clientId{A}. If the entry doesn't exist, the flow terminates with error.
   \item \Server{} executes \ref{update_server_bontact_book_flow:one}.
  \end{enumerate}
  \item If the operation is {\em Update}
  \begin{enumerate}
   \item \Server{} updates the graph.
  \end{enumerate}
 \end{enumerate}
\end{enumerate}

\paragraph{Implementation Note} If one of the add and delete operations for the comparison identities fails, the flow terminates with error. This means that the graph update steps 
must be triggered after the flow finishes, to avoid having generated spurious results if the DB is rolled back.

\subsection{Send Contact Card}
\label{send_contact_card_flow}
This flow is executed when \user{A} wants to share their contact card with \user{B}.

\paragraph{Participants} \User{A}, \User{B}, \Client{A}, \Client{B}, \Server{}

\paragraph{Outputs}
If the flow succeeds, \Client{A} has the contact card of \Client{B}.

\paragraph{Other Variables}
\SpecialItem
\begin{description}
 \item[U : HashBuffer] The blinded identity of \User{B} known to \Client{A}.
 \item[\privateKeys{} : PrivateKeys] The private keys of the client.
 \item[\publicKeys{} : PublicKeys] The public keys of the client.
 \item[\signedData{} : SignatureBuffer] The signed data to be sent from a client.
 \item[\encryptedData{} : EncryptedBuffer] The encrypted contact card.
 \item[\userId{A} : Id] The user id of \User{A}.
 \item[\userId{B} : Id] The user id of \User{B}.
 \item[\contactCard{B}: ContactCard] The contact card of \User{B}.
\end{description}

\paragraph{Flow}

\begin{enumerate}
 \item \User{A} requests the contact card of \User{B}. This may happen by \User{A} selecting \User{B} from the contact book of \Client{A} or by \Server{} sending a suggestion to 
\Client{A} for \User{B} in flow \ref{suggest_new_contact_flow}.
 \item \Client{A} populates $U$ by picking up the primary identity for \User{B} from the \longProduct{} contact book.
 \begin{enumerate}
  \item If there is none, the flow terminates with error. Normally, this case is never reached, because contacts for which no identity is known are neither in \longProduct{} 
contact book or in \Server{} contact book, but race conditions may arise.
  \item If there are more than one, a resolution is provided by the native contact book. If there is no native contact book or there is a draw, one is picked arbitrarily.
 \end{enumerate}
 \item \Client{A} populates \privateKeys{} from local storage.
 \item \Client{A} populates \signedData{} from \signingKey{} and $U$ and \userId{B}.
 \item \Client{A} sends \signedData{} to \Server{}.
 \item \Server{} populates \publicKeys{} for \Client{A}.
 \item \Server{} verifies \signedData{}.
 \item \Server{} retrieves from UserSet all clients for \userId{B}.
\end{enumerate}
For the rest of the flow, we assume that there is only one client for \User{B}. If there are more, \Server{} executes \ref{send_contact_card_flow:one} for all clients for 
\userId{B} and the rest of the flow is executed for each one that responds. If more than one contact cards reach \Client{A}, they are merged in the native and \longProduct{} 
contact books.
\begin{enumerate}[resume]
 \item\label{send_contact_card_flow:one} \Server{} sends to \Client{B} \userId{A}, \clientId{A}, \encryptionKey{} for all clients for \userId{A}.
 \item \Client{B} requests permission from \User{B} to send the contact card.
 \begin{enumerate}
  \item If \User{B} declines, the flow terminates.
 \end{enumerate}
 \item \User{B} grants permission for a non-empty subset of identities.
 \item \Client{B} populates \contactCard{B} from the selected identities.
 \item \Client{B} populates \encryptedData{} from \encryptionKey{} and \contactCard{B}, for all \encryptionKey{}s that have been received.
 \item \Client{B} populates \privateKeys{}.
 \item \Client{B} populates \signedData{} from \signingKey{} and \encryptedData{}, \userId{A}, and \clientId{A}.
 \item \Client{B} sends to \Server{} \signedData{}.
 \item \Server{} populates \publicKeys{} for \Client{B}.
 \item \Server{} verifies \signedData{}.
 \item For each client of \userId{A} for which \encryptedData{} has been received, let it be \Client{A}'
 \begin{enumerate}
  \item \Server{} sends to \Client{A}' \signedData{}, \verificationKey{}.
  \item \Client{A}' verifies \signedData{} with \verificationKey{}.
  \item \Client{A}' decrypts \signedData{} to populate \contactCard{B}.
  \item \Client{A}' merges \contactCard{B} into the contact book entry for \User{B}.
 \end{enumerate}
\end{enumerate}

\subsection{Suggest New Contact}
\label{suggest_new_contact_flow}
This flow is executed when \server{} believes that \user{A} should know \user{B}. The symmetrical flow is executed with \user{B} and \user{A} exchanging roles.

\paragraph{Participants} \User{A}, \Client{A}, \Server{}

\paragraph{Input}
\SpecialItem
\begin{description}
 \item[\userIdSimple{A} : Id] The user id of the receiver of the contact suggestion.
 \item[\userIdSimple{B} : Id] The user id of the suggested contact.
\end{description}

\paragraph{Other Variables}
\SpecialItem
\begin{description}
 \item[name : string] The name of \User{A}, retrieved from the contact card.
\end{description}

\paragraph{Flow}

\begin{enumerate}
 \item \Server{} checks if $U(B)$ is in UserSet.SentRequests for \userIdSimple{A}.
 \begin{enumerate}
  \item If it is, the flow terminates.
 \end{enumerate}
 \item For all \Client{A} in UserSet for \userIdSimple{A}
 \begin{enumerate}
  \item \Server{} sends \userIdSimple{B}, $U(B)$, \encryptionKey{} for all clients for \userIdSimple{B}.
  \item \Client{A} requests permission from \User{A} to send a contact request to \User{B}.
  \item \User{A} grants permission.
  \item \Client{A} populates $name$ from the contact card.
  \item \Client{A} encrypts and signs $name$ and sends it to \Server{} with signed $U(B)$.
  \item \Server{} verifies the signature and checks if that's the first reply from \Client{A} for $U(B)$ in UserSet.SentRequests.
  \begin{enumerate}
   \item If it is not, the flow terminates.
   \item If it is, it enters it in UserSet.SentRequests.
  \end{enumerate}
 \end{enumerate}
 \item \Server{} forwards the request to all \Client{B}s.
 \item \Client{B} requests permission from \User{B} to execute the Send Contact Card \ref{send_contact_card_flow} flow.
 \item \User{B} grants permission.
 \item \Client{B} executes the Send Contact Card \ref{send_contact_card_flow} flow with \User{A}.
\end{enumerate}

\subsection{Confirm Identity}
\label{confirm_identity_flow}
This flow is executed when \user{A} wants \user{B} to confirm one or more of \user{A}'s identities.

\paragraph{Participants} \User{A}, \Client{A}, \User{B}, \Client{B}, \Server{}

\paragraph{Input}
\SpecialItem
\begin{description}
 \item[]
\end{description}

\paragraph{Other Variables}
\SpecialItem
\begin{description}
 \item[]
\end{description}

\paragraph{Flow}

\begin{enumerate}
 \item
\end{enumerate}

\subsection{Confirm Keys}
\label{confirm_keys_flow}
This flow is executed when for preventing man-in-the-middle attacks. \user{A} and \user{B} confirm that the keys for each other's clients are what \Server{} claims them to be 
through a separate channel.

\paragraph{Participants} \User{A}, \Client{A}, \User{B}, \Client{B}

\paragraph{Input}
\SpecialItem
\begin{description}
 \item[]
\end{description}

\paragraph{Other Variables}
\SpecialItem
\begin{description}
 \item[]
\end{description}

\paragraph{Flow}

\begin{enumerate}
 \item
\end{enumerate}

\subsection{Register New Client}
\label{register_new_client_flow}
This flow is executed when \user{A} with already registered \client{$A_1$} wants to register a new \client{$A_2$}. It doesn't cover the case where \client{$A_1$} is inaccessible. 
In this case \ref{register_identity_flow} needs to be executed.

\paragraph{Participants} \User{A}, \Client{$A_1$}, \Client{$A_2$}

\paragraph{Input}
\SpecialItem
\begin{description}
 \item[]
\end{description}

\paragraph{Other Variables}
\SpecialItem
\begin{description}
 \item[]
\end{description}

\paragraph{Flow}

\begin{enumerate}
 \item
\end{enumerate}

\bibliography{spec}
\bibliographystyle{plain}

\end{document}
