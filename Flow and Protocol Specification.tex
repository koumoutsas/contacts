\documentclass[a4paper,10pt,draft]{article}

\newcommand{\yannis}{Yannis Ioannidis}
\newcommand{\longProduct}{Contacts Discovery}
\newcommand{\shortProduct}{CD}
\newcommand{\client}{{\em Client}}
\newcommand{\server}{{\em Server}}


\newcommand{\signedUserData}{\emph{signedUserData}}
\newcommand{\newPublicKey}{\emph{newPublicKey}}
\newcommand{\newPrivateKey}{\emph{newPrivateKey}}
\newcommand{\code}{\emph{code}}
\newcommand{\serverVerified}{\emph{serverVerified}}
\newcommand{\handle}{\emph{handle}}
\newcommand{\personalData}{\emph{personalData}}
\newcommand{\registrationCode}{\emph{registrationCode}}
\newcommand{\userAgent}{\emph{userAgent}}
\newcommand{\privateKey}{\emph{privateKey}}
\newcommand{\publicKey}{\emph{publicKey}}
\newcommand{\userId}{\emph{userId}}
\newcommand{\clientId}{\emph{clientId}}
\newcommand{\loggedIn}{\emph{loggedIn}}
\newcommand{\fixedString}{\emph{fixedString}}
\newcommand{\removeUser}{\emph{removeUser}}
\newcommand{\newHandle}{\emph{newHandle}}

\title{\longProduct{} System Flows and Protocols Specification}

\begin{document}

\maketitle
\tableofcontents

\section{Purpose}
This document specifies the protocols and flows executed for the \longProduct{} system (\shortProduct{}) between the \server{} and \client{} entities. The protocols cover contact 
management and administration flows (account creation, account deletion, etc).

\section{References}
For the project overview, see~\cite{project_spec}. For the supporting data structures, see~\cite{data_structure_spec}. For details about security and cryptography, 
see~\cite{crypto_spec}. For the client UI elements specification, see~\cite{UI_spec}.

\section{Definitions, Nomenclature and Conventions}

\subsection{Entities}
We refer to three entities in this document. These represent different participants in a flow. A short explanation on each one follows. For more detail on their role in the 
system, see~\cite{project_spec}. Entities are denoted in all uppercase.

\begin{description}
 \item[\Server{}] includes the central business logic, the database and all interfaces to third-party systems for, among others, sending out notifications and mail.
 \item[\Client{}] is an abstraction of a user agent. It usually refers to the mobile app that a customer uses, but it can extend to desktop applications, test clients, etc.
 \item[\User{}] This is a special entity that refers to the human using \Client{}. It is used for flow steps that require interaction between the client user agent and the human 
user. This entity is not used for protocol specifications. It is used in flow specifications to disambiguate the origin and target of some steps.
\end{description}

\subsection{Notation for Variables}
\label{sec:variable_notation}
Because in each flow or protocol multiple participants are involved, variables are denoted with a hierarchical name, with each level denoted as a dot. For example, variable $X$ of 
entity $Y$ is denoted as $Y.X$. Inputs and outputs are considered variables.

Variables are declared as
\begin{center}
$\langle Entity\rangle.\langle Name\rangle:\langle type\rangle\{[qualification_1, qualification_2, \dots]\}$.
\end{center}

For example, let $bar$ be an unsigned integer variable of entity $foo$, that takes values in the range $[10,100]$ and has default value 5. The declaration would look like:

\begin{center}
 foo.bar:unsigned integer[range:[10,100],default value:5].
\end{center}

The qualifications parts is optional and its syntax is that of a $key:value$ pair.

A data structure $D$ may appear without prior definition, when it has been specified in~\cite{data_structure_spec}. For specifications of types related to security and 
cryptography, such as registration codes and RSA keys, see~\cite{crypto_spec}.

All variable names are small camelCase.

\subsection{Flow Specifications Organization and 
Format}\label{sec:flow_spec_def}
Each flow description is organized in the following sections:

\begin{description}
 \item[Description] What the flow does.
 \item[Participants] A list of participating entities.
 \item[Inputs] A list of the inputs to the flow. See~\ref{sec:variable_notation} for the variable syntax.
 \item[Outputs] A list of the expected outputs to the flow. The syntax is the same as for inputs. Their description may be simplified when their meaning can be inferred from a 
more informal description.
 \item[Other variables] Any other variables that are instantiated during the flow.
 \item[Local definitions] Definitions of functions that are local to the flow. We provide in~\ref{sec:definitions} some common global definitions, but there may be local functions 
that only make sense in the context of a flow.
 \item[Steps] An enumeration of the steps. The first word in each step is the initiating entity, meaning the entity that sends a message in this step.
\end{description}

\subsection{Protocol Specifications Organization and Format}

\subsection{Definitions}\label{sec:definitions}

\begin{description}
 \item[Public/Private Key] The public and private keys for entity $A$ are denoted as $K_{public}^A$ and $K_{private}^A$. The encryption of message $m$ for entity $A$ with key 
$K_{public}^A$ is denoted as $E(A,m)$. The decryption of encrypted message $d$ by entity $A$ with key $K_{private}^A$ is denoted as $D(A,d)$. The identity $m=D(A,E(A,m))$ holds.
 \item[Signature and Verification] The signing and verification keys for entity $A$ are denoted as $K_{sign}^A$ and $K_{verify}^A$. The signature of message $m$ by entity $A$ is 
denoted as $S(A,m)$. The verification of signature $s$ produced by $A$ is denoted as $V(A,s)$. The identity $m=V(A,S(A,m))$ holds.

Note that signing a message $m$ involves signing a hash of $m$ and accompanying the signature with $m$. This means that when \Client{} signs a message $m$ and sends the signature 
$s$ to \Server{}, \Server{} knows from $s$ the details of the message, like the client id, without having to decrypt $s$ first. See~\cite{crypto_spec} for more details.
\end{description}

\subsection{Alternative Steps}
Some parts of the flow may have alternative specifications. These alternatives originate from usability concerns or uncertainty about hardware components, which may require 
multiple ways to effect the same action. Alternative steps are defined after the primary specification.

Alternative flow steps should be considered as future extensions. Initial implementation follows the primary flow specification, but it should take into account that alternative 
steps may replace the primary ones or that extensions of the implementation may incorporate multiple alternatives. Details about implementation priority and plans for these 
alternatives can be found in accompanying footnotes.

\section{Flows}

\subsection{Implicit Flow Branches}
Some flow branches are repeated and are followed under common triggers. To avoid cluttering the main flows, we define them here.

\subsubsection{Registration Code Verification}
When a step states that \Server{} checks the registration code, the following steps are implied:

\begin{enumerate}
 \item \Server{} checks the registration code in $RegistrationCodeMap$, with the key being \userId{} and \clientId{}.
 \begin{enumerate}
  \item If the checks succeeds, \Server{} removes the entry from\\$RegistrationCodeMap$.
  \item If the checks fails, \Server{} sends error to the entity that requested the check and terminates the flow.
 \end{enumerate}
\end{enumerate}

\subsubsection{Flow Termination with Error on Failed Commit}
Steps which state that entity $X$ commits $Y$ to the DB imply a flow branch that is followed when the commit operation fails. In this case, $X$ terminates the flow with an error. 
In most cases, error flows are inferable from other error branches. When this is not possible, the branch is declared explicitly. 

\subsubsection{\Server{} Errors}
All flows involving \Server{} and another entity may fail for various reasons, such as \userId{} or \clientId{} being unknown or the signature verification failing. For security, 
\Server{} does not reveal the specific reason the operation failed. Therefore, when, for example, a registration code verification fails, the message from \Server{} to \Client{} 
does not differentiate between all the reasons the verification may fail.

\bibliography{spec}
\bibliographystyle{plain}

\end{document}
