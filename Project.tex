\documentclass[a4paper, twoside, 10pt]{article}
\usepackage[utf8]{inputenc}
\usepackage{verbatimbox}
\usepackage{hyperref}
\usepackage{tabularx}

\usepackage[utf8]{inputenc}
\usepackage[american]{babel}
\usepackage{fontenc}
\usepackage{graphicx}
\usepackage{verbatimbox}
\usepackage{hyperref}
\usepackage{paralist}
\usepackage{enumerate}
\usepackage{amsmath}
\usepackage{enumitem}
\usepackage{gitinfo2}
\usepackage{xifthen}

\newcommand{\parenthesize}[2]{{\em  #1\ifthenelse{\isempty{#2}}{}{(#2)}}}
\newcommand{\longProduct}{Contacts Discovery}
\newcommand{\shortProduct}{CD}
\newcommand{\commercialName}{CD}
\newcommand{\client}[1][]{\parenthesize{Client}{#1}}
\newcommand{\server}{{\em Server}}
\newcommand{\user}[1][]{\parenthesize{User}{#1}}
\newcommand{\Server}{{\em \MakeUppercase{server}}}
\newcommand{\Client}[1][]{\parenthesize{\MakeUppercase{client}}{#1}}
\newcommand{\User}[1][]{\parenthesize{\MakeUppercase{user}}{#1}}

\author{\gitAuthorName{}\\\gitAuthorEmail{}}
\date{\gitAuthorDate{}\\Version \gitAbbrevHash{} \gitVtag{}}

\let\Item\item
\newcommand\SpecialItem{\renewcommand\item[1][]{\Item[\textbullet~\bfseries##1]}
}
\renewcommand\enddescription{\endlist\global\let\item\Item}


\title{\longProduct{} System}
\author{\yannis}

\setcounter{secnumdepth}{5}

\begin{document}
\maketitle

\begin{abstract}
This document describes a system for contacts discovery. We outline the idea, its business case and system architecture.
\end{abstract}

\tableofcontents

\section{Introduction}

The purpose of the \longProduct{} system (\commercialName{}) is to allow people that have a common friend exchange contact information on their smartphones and keep this contact 
information up to date.

There are two points of emphasis: automated contact information exchange and maintenance, and privacy. On one hand, users are able to receive contact information without manually 
entering the information in their contacts books. On the other hand, each user's information is transmitted to someone else only with their explicit consent and the server doesn't 
have access to the users' contact or other information.

\section{Purpose and Use Cases}

\commercialName{} is meant as a complement to smartphone contact books that allows ease of adding new entries and keeping existing entries up to date. Some use cases are:

\begin{itemize}
  \item Alice just moved to a new city where she knows only Bob. She meets Cloe, a friend of Bob's. Alice, Bob and Cloe all use \commercialName{}. Alice doesn't have to write 
down Cloe's phone number, email or Facebook handle. She can search for Cloe in her two-degrees of separation radius and make a request through \commercialName{}. Cloe accepts and 
they exchange information in a matter of seconds.
  \item Alice knows Bob and Cloe, who both know Dave, who knows Ellen who turns out to be a colleague of Alice. Alice and Dave have met, but they haven't had the opportunity to 
exchange phone numbers. \commercialName{} has detected that Alice and Dave have a high probability to know each other and has them as each other suggested connection. Dave asks 
Alice through the app for her phone number. Alice accepts and they exchange numbers.
  \item Alice is the CEO of her company. She goes into a meeting with Bob, her CFO. Bob has arranged this meeting with Cloe and Dave, the founders of a startup, which were brought 
into contact with Bob through Ellen and Fred who work for a bank. All six of them enter the meeting and never take out their business cards for the traditional card swap frenzy, 
because they use \commercialName{}. \commercialName{} has exchanged information for everybody already, with their picture if available, so they can keep track of who is who. No 
one has to add a card to a Rolodex holding 1000 cards or lose one in the laundry or Google to connect a face with a name.
  \item Alice has quit her company and with that her business phone number. She got a new number, but now she needs to notify everyone. Except that \commercialName{} has detected 
the new number and has updated all her friends and family's contact books with the new number.
\end{itemize}

\subsection{Privacy}

While all the above use cases involve the exchange of private and sensitive information, the servers of \commercialName{} do not see or store any of it. There is a perfect privacy 
layer around all exchanges. The servers have two functions:

\begin{itemize}
 \item Transports of messages. These messages are encrypted and only the participating users can see their content.
 \item They detect which users have a high probability that they know each other and suggest them as contacts or search them in the two degrees of separation radius. However, no 
user app ever sends their contact information to the server, not even their name. All the information matching is done on cryptographically secure blinded representations.
\end{itemize}

As a result the service itself knows nothing about the users. The users can even check what leaves their clients and the client code is open-sourced, so that security experts can 
verify that no information leaks from the client.

\section{Competitive Analysis}

\section{References}
For detailed system design, see~\cite{data_structure_spec},~\cite{flow_spec},~\cite{UI_spec},~\cite{crypto_spec}.

\bibliography{spec,project}
\bibliographystyle{plain}

\end{document}
